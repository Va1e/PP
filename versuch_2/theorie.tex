\section{Theoretische Betrachtung}
\subsection{Durchlassfilter}
Ein Durchlassfilter ist eine Reihenschaltung aus Kondensator, Spule und Widerstand. \aref{plan:durchlass}

Mit den Kirchhoffschen Regeln und 
$
U=Z\cdot I
$ 
gilt \footnote{\cite[151 ff.]{demtroder2009experimentalphysik}}
\begin{equation}
U_a=\frac{R}{R+i\left(\omega L-\frac{1}{\omega C}\right)}\cdot U_e\\
\end{equation}
\begin{equation}
\Rightarrow\left| U_a \right| = \frac{R}{\sqrt{R^2+\left(\omega L - \frac{1}{\omega C}\right)}}\cdot \left| U_e \right|
\end{equation}
Bei der Frequenz
\begin{equation}
\omega=\omega_R=\frac{1}{\sqrt{L\cdot C}}
\end{equation}
wird
$
\left|U_a \right| = \left|U_e \right|
$
, d.h. die Wechselspannung $U_e(\omega_R)$ wird vollständig durchgelassen, während alle anderen Frequenzen abgeschwächt werden.
  Setzt man nun
\begin{equation}
\frac{\left|U_a \right|}{\left|U_e \right|}=\frac{1}{\sqrt{2}}
\end{equation}
und löst die quadratische Gleichung, ergibt sich die Bedingung
\begin{equation}
\omega_{1,2}=\pm \frac{R}{2L}+\sqrt{\frac{R^2}{4L^2}+\omega_R^2}
\end{equation}
Berechnet man nun die Frequenzbreite
$
\Delta \omega=\omega_1-\omega_2
$
, ergibt sich
\begin{equation}
\Delta \omega=\frac{R}{L}
\end{equation}
, d.h. bei kleineren Widerständen ergeben sich 'schärfere' Peaks.
\paragraph{}Bei dieser Betrachtung wird angenommen, dass der Ohmsche Widerstand der Spule vernachlässigbar klein ist. Dies in der Praxis jedoch nicht immer der Fall.
Wenn die Spule einen Widerstand $R_L$ besitzt, ändert sich Gleichung (2) zu
\begin{equation}
\Rightarrow\left| U_a \right| = \frac{R}{\sqrt{(R+R_L)^2+\left(\omega L - \frac{1}{\omega C}\right)}}\cdot \left| U_e \right|
\label{form:durchlass}
\end{equation}
also wird die Wechselspannung $U_e$ selbst bei der Resonanzfrequenz nicht vollständig durchgelassen.
\begin{figure}
\centering
\begin{circuitikz}[european voltages]
\draw
  (0,0) to [short, -o] (6,0)
  (6,0) to [open, l_=$U_a$] (6,4) %Ausgangsspannung
  %to [voltmeter, l_=$U_a$] (6,4) % Messung
  (6,4) to [short, o-] (5,4) 

  (0,0) to [csV, l=$U_e$] (0,4) % Eingangsspannung
  %to [short] (1,4)
  to [C, l=$C$] (2.5,4) % Kondensator
  to [L, l=$L$] (4.5,4) % Spule
  to [short] (5,4)
  to [R, l_=$R_m$] (5,0); % Messwiderstand

\end{circuitikz}
\caption{Schaltbild eines Durchlassfilters}
\label{plan:durchlass}
\end{figure}

\subsection{Sperrfilter}
Beim Sperrfilter werden die Spule und der Kondensator nun parallel geschaltet. (siehe Abb. \ref{plan:sperr} auf Seite \pageref{plan:sperr})

Analog zum Durchlassfilter ergibt sich nun
\begin{equation}
U_a=\frac{R}{R-i \frac{1}{\omega C - \frac{1}{\omega L}}}\cdot U_e\\
\end{equation}
\begin{equation}
\Rightarrow\left| U_a \right| = \frac{R}{\sqrt{R^2+\frac{1}{(\omega C - \frac{1}{\omega L})^2}}}\cdot \left| U_e \right|
\end{equation}
Bei der Resonanzfrequenz (3) geht nun $\left| U_a \right|\longrightarrow 0$, d.h. die Frequenz $\omega_R$ wird blockiert, während andere Frequenzen durchgelassen werden.


\paragraph{}Berechnet man hier die Frequenzbreite analog zu (4) und (5) ergibt sich:

\begin{equation}
\Delta\omega = \frac{1}{RC\sqrt{1-(1-\frac{1}{\sqrt{2}})^2}}
\end{equation}

Hier ergeben größere Widerstände 'schärfere' Peaks.  
\newpage Auch hier kann man der Spule einen Widerstand $R_L$ zuweisen, außerdem wurde ein nicht weiter spezifizierter zusätzlicher Widerstand $R_{zus}$ eingefügt, der die Widerstände der Steckverbindungen darstellt, s iehe Abb. \ref{plan:sperr-R_L} auf Seite \pageref{plan:sperr-R_L}.
\begin{figure}
\centering
\begin{circuitikz}
\draw
  (0,1) to [short, -o] (6,1)
  (6,1) to [open, l_=$U_a$] (6,4) %Ausgangsspannung
  (6,4) to [short, o-] (5,4) 

  (0,1) to [csV, l=$U_e$] (0,4) % Eingangsspannung
  to [short, ] (1,4)
  (1,5) to [short] (1,3)
  to [C, l_=$C$] (4,3) % Kondensator
  to [short] (4,5)
  to [L, l=$L$] (1,5) % Spule
  (4,4) to [short] (5,4) 
  to [R, l_=$R_m$] (5,1); % Messwiderstand
\end{circuitikz}
\caption{Schaltbild eines Sperrfilters}
\label{plan:sperr}
\end{figure}


\paragraph{} Der Gesamtwiderstand der Parallelschaltung aus Spule und Kondensator ist dann:

\begin{equation}
Z_{SK} = \frac{1}{\frac{1}{R_L+i\omega L}+i\omega C}
\end{equation}\\

Der Gesamtwiderstand der Schaltung kann nun geschrieben werden als

\begin{equation}
Z_{SK}+R_{zus}+R = \frac{\gamma R_L - i\omega\gamma[C\gamma-L]}{R_L^2+\omega^2[C\gamma-L]^2} + R_{ges}
\end{equation}\\

mit $\gamma=R_L^2+\omega^2L^2$ und $R_{ges}=R+R_{zus}$\\

Analoges Vorgehen zu (1) und (2) liefert schließlich:

\begin{equation}
\frac{\left|U_a \right|}{\left|U_e \right|} = \frac{R \beta}{\sqrt{\left(\gamma R_L + R_{ges}\beta\right)^2+ \gamma^2 \epsilon^2}}
\end{equation}

mit $\gamma=R_L^2+\omega^2L^2$, $\epsilon=\omega[C\gamma-L]$, $\beta=R_L^2 + \epsilon^2$\\

Einsetzen der Substitutionen und Ausschreiben liefert dann folgende Gleichung, die für die Theoriekurven in Abb. \ref{plot:sperr} verwendet wurde.

\footnotesize
\begin{align}
&\left| \frac{U_a}{U_e} \right| = \label{form:sperr} \\
&\frac{R\left(R^2_L+\omega ^2\left[C\left(R^2_L+\omega ^2 L^2 \right) - L\right] ^2 \right)}{\sqrt{ \left( R^2_L + \omega ^2L^2\right) R_L+R_{ges}\left\{ R_L^2+\omega ^2\left[ C\left( R_L^2+\omega ^2L^2\right) -L\right] ^2\right\}^2+\left(R^2_L+\omega^2L^2\right)^2\omega^2\left[C\left(R^2_L+\omega^2L^2\right)-L\right]}}\nonumber
\end{align}
\normalsize
\begin{figure}
\centering
\begin{circuitikz}
\draw
  (0,1) to [short, -o] (7,1)
  (6,1) to [open, l_=$U_a$] (7,4) %Ausgangsspannung
  (7,4) to [short, o-] (6,4) 

  (0,1) to [csV, l=$U_e$] (0,4) % Eingangsspannung
  to [short, ] (1,4)
  (1,5) to [short] (1,3)
  to [C, l_=$C$] (4,3) % Kondensator
  to [short] (4,5)
  to [R, l=$R_L$] (2.5,5) %Widerstand der Spule
  to [L, l=$L$] (1,5) % Spule
  (4,4) to [R,l=$R_{zus}$] (6,4) 
  to [R, l_=$R_m$] (6,1); % Messwiderstand
\end{circuitikz}
\caption{Position der zusätzlich angenommenen Widerstände}
\label{plan:sperr-R_L}
\end{figure}