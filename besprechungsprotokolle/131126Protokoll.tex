\documentclass[
]{scrartcl}

%Deutsche Sprachunterstützung
\usepackage[utf8]{inputenc}
\usepackage[ngerman]{babel}
\usepackage{marvosym}
\DeclareUnicodeCharacter{20AC}{\EUR}

%Für das Einbinden von Bildern
%\usepackage{graphicx}

%Tabellen
\usepackage{array}

%Tabellen automatisch schoener
\usepackage{booktabs}

%Caption
\usepackage{caption}
\usepackage{subcaption}

%Formeln
\usepackage{mathtools}
\usepackage{amsmath}
\usepackage{amssymb}
\usepackage{amstext}
\usepackage{dsfont}

%\usepackage{mnsymbol}

%Vectorpfeile schöner
\usepackage{esvect}

%Formatierung
\usepackage[T1]{fontenc}
\usepackage{lmodern}
\usepackage{microtype}
%\usepackage[german=guillemets]{csquotes}

%Formatierungsanweisungen
\newcommand{\wichtig}[1]{\underline{\large{#1}}}
\newcommand{\aref}[1]{(s.Abb. \ref{#1})}
\newcommand{\R}{\mathbb{R}}
\newcommand{\K}{\mathbb{K}}
\newcommand{\C}{\mathbb{C}}
\begin{document}
\title{7. Besprechungsprotokoll PPG8}
\date{26.11.2013}
\maketitle

\paragraph{Anwesende:}
Udo Beier, Leon Br"uckner, Valentin Olpp, Marco Zech, Sebastian Ziegler, Domenico Tiziani


\section{Status des Projekts}

\subsection{Bisheriger Verlauf und momentaner Status}
Die bei der Messung mit Cassy entstandenen Daten waren nicht auswertbar, da sich die gemessenen Spannungsdifferenzen kaum vom Grundrauschen der zur Messung benutzten Photodiode unterschieden.

\subsection{Weiteres Vorgehen}
Das Experiment wird erneut durchgeführt, nur werden zur Auswertung nun die innerhalb von 10 Sekunden durchlaufenen Maxima per Hand gezählt, um hieraus einen zeitabhängigen Längenausdehnungskoeffizienten zu ermitteln.

\subsection{Arbeitsaufteilung}
Udo Beier: Zählt die Interferenzringe an der Wand; übernimmt die Auswertung. \\
Valentin Olpp: Überwachung des Cassy-Systems während der Messung. \\
Sebastian Ziegler: Arbeitet am Theorieteil des Protokolls. \\
Leon Brückner, Marco Zech: Sorgen während des Versuches dafür, dass möglichst wenige Personen am Raum vorbeilaufen und so den empfindlichen Versuchsaufbau beeinflussen. Leon Brückner verfasst weiter den Durchführungsteil zu diesem Versuch. 
\end{document}
