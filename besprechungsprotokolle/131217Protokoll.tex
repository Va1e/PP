\documentclass[
]{scrartcl}

%Deutsche Sprachunterstützung
\usepackage[utf8]{inputenc}
\usepackage[ngerman]{babel}
\usepackage{marvosym}
\DeclareUnicodeCharacter{20AC}{\EUR}

%Für das Einbinden von Bildern
%\usepackage{graphicx}

%Tabellen
\usepackage{array}

%Tabellen automatisch schoener
\usepackage{booktabs}

%Caption
\usepackage{caption}
\usepackage{subcaption}

%Formeln
\usepackage{mathtools}
\usepackage{amsmath}
\usepackage{amssymb}
\usepackage{amstext}
\usepackage{dsfont}

%\usepackage{mnsymbol}

%Vectorpfeile schöner
\usepackage{esvect}

%Formatierung
\usepackage[T1]{fontenc}
\usepackage{lmodern}
\usepackage{microtype}
%\usepackage[german=guillemets]{csquotes}

%Formatierungsanweisungen
\newcommand{\wichtig}[1]{\underline{\large{#1}}}
\newcommand{\aref}[1]{(s.Abb. \ref{#1})}
\newcommand{\R}{\mathbb{R}}
\newcommand{\K}{\mathbb{K}}
\newcommand{\C}{\mathbb{C}}
\begin{document}

\title{Besprechungs-Protokoll Gruppentreffen PPG8-10}
\date{17. Dezember 2013}
\maketitle

\paragraph*{Anwesende}
Udo Beier,
Leon Brückner,
Valentin Olpp,
Marco Zech,
Sebastian Ziegler,
Domenico Tiziani


\section{Status des Projekts}

\subsection{Zweiter Versuch}
Das Protokoll wurde vom Tutor korrigiert und nach einigen Verbesserungen am Montag bei Fr. Anton abgegben.

\subsection{Dritter Versuch}
Die Messungen am Doppelpendel sind abgeschlossen.

\subsection{Vierter Versuch}
Als Vorbereitung auf den vierten Versuch (Test verschiedener Strömungsmesstechniken) wurden bislang theoretische Betrachtungen zum Bau einer Brandelsonde/Kielsonde bzw. eines Pitotrohres angestellt.

\section{Weiteres Vorgehen}
Demnächst werden Marco Zech und Udo Beier die theoretischen Betrachtungen, Sebastian Ziegler die Auswertung, Valentin Olpp die Versuchsdurchführung und Leon Brückner das Vorwort zum dritten Versuch verfassen.
Außerdem wird der vierte Versuch vorbereitet.
\end{document}
