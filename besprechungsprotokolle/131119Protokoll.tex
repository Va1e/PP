\documentclass[
]{scrartcl}

%Deutsche Sprachunterstützung
\usepackage[utf8]{inputenc}
\usepackage[ngerman]{babel}
\usepackage{marvosym}
\DeclareUnicodeCharacter{20AC}{\EUR}

%Für das Einbinden von Bildern
%\usepackage{graphicx}

%Tabellen
\usepackage{array}

%Tabellen automatisch schoener
\usepackage{booktabs}

%Formatierung
\usepackage[T1]{fontenc}
\usepackage{lmodern}
\usepackage{microtype}
\usepackage{circuitikz}
%\usepackage[german=guillemets]{csquotes}
%Formatierungsanweisungen
\newcommand{\wichtig}[1]{\underline{\large{#1}}}
\newcommand{\aref}[1]{(s.Abb. \ref{#1})}
\newcommand{\R}{\mathbb{R}}
\newcommand{\K}{\mathbb{K}}
\newcommand{\C}{\mathbb{C}}


\begin{document}
\title{6. Besprechungsprotokoll PPG8}
\date{19.11.2013}
\maketitle

\paragraph{Anwesende:}
Udo Beier, Leon Br"uckner, Valentin Olpp, Marco Zech, Sebastian Ziegler, Domenico Tiziani


\section{Status des Projekts}

\subsection{Momentaner Status}
Der Versuchsaufbau ist fast komplett, beim Bewegen des Verfahrtischs sind schon Schwingungen auf dem Oszilloskop zu erkennen, ein klares Problem sind aber noch die sehr geringen Ausschläge und das starke Rauschen.

\subsection{Weiteres Vorgehen}
Ein Gespräch mit Herrn Sondermann hat ergeben, dass die geringen Ausschl"age vor allem in der sehr starken Aufweitung des Strahls begr"undet sind. Ein neuer Aufbau, der eine Art 'Teleskop' zum Aufweiten des Strahls beinhaltet, soll h"oheren Kontrast und damit bessere Messergebnisse liefern.
In dieser Woche soll also der Aufbau des Interferometers verbessert werden sowie die eigentliche Messung, die Längenänderung eines Metallstabs mit der Temperatur, aufgebaut und durchgeführt werden.

\subsection{Arbeitsaufteilung}
Udo Beier: Aufbau und Justierung der neuen Versuchsanordnung
\\
Leon Br"uckner: Aufbau der Messung f"ur die L"angen"anderung, Besorgung von fl"ussigem Stickstoff zur Kühlung des Stabs
\\
Valentin Olpp: Aufbau und Justierung der neuen Versuchsanordnung
\\
Marco Zech: Aufbau und Inbetriebnahme des CASSY-Systems
\\
Sebastian Ziegler: Justierung der neuen Versuchsanordnung, Schreiben des Theorieteils f"ur das Protokoll


\end{document}