\section{Vorwort}
Das Pitot Rohr ist ein im 18. Jahrhundert erfundenes Strömungsmessgerät, das nach dem Wasserbauingenieur Henri Pitot, der sich mit Strömungen in Kanälen auseinandersetzte, benannt. Besonders in der Luftfahrt und im Motorsport ist die Messung der Strömungsgeschwindigkeit von besonderer Bedeutung. Um die Messungen zu verbessern wurden unterschiedliche Sondenformen konzipiert, dazu gehören kombinierte Messonden für Stau- und Statischendruck und andere Formen von Messköpfen um unterschiedliche Anstellwinkel zu ermöglichen. Im Flugzeugbau werden heute sowohl mechanische als auch elektronische Varianten der Messrohre eingesetzt. Im folgenden wird eine Sonde mechanischer Bauform betrachtet.
